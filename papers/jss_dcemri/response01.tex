\documentclass[11pt]{article}
\usepackage{a4wide}
\usepackage[usenames]{color}
\usepackage[authoryear,round]{natbib}
\bibliographystyle{plainnat}
\pagestyle{plain}

\begin{document}

% Title

\begin{center}
{\Large \textbf{Quantitative Analysis of Dynamic Contrast-Enhanced and Diffusion-Weighted Magnetic Resonance Imaging for Oncology in \textsf{R}}}

\bigskip

March~1, 2011

\end{center}

\section*{Associate Editor}

\subsection*{General Comments}

\begin{itemize}

\item Generating the pdf document from Sweave includes information on
  different machines (where the creation has been done).  Thus the
  audit.trail in Fig.10 should be edit slightly in the final version
  (working directory).

  \textbf{TO DO!}

\end{itemize}

\subsection*{Specific Comments}

\begin{itemize}

\item Can the grey scale contrast be enhanced in the images?

  \textbf{TO DO!}

\end{itemize}

\section*{Reviewer \#1}

We would like to thank the reviewer for all his/her comments and
suggestions.

\subsection*{Minor Remarks}

\begin{enumerate}

\item A colour scale would aid interpretation of the various
  quantitative figures.

  \textbf{TO DO!}

\item The authors are not entirely consistent about the use (or not)
  of a subscript in relaxation time variables such as T1; e.g. at the
  bottom of page 2, ``T1'' is written both with and without a
  subscript.

  We have adopted the notation ``T1'' and ``T2'' throughout the
  manuscript.

\item A little more explanation of the final aim of the processing
  (i.e.  maps of estimated kinetic parameters, etc.), and an
  indication of the reasons that they are important in oncology, would
  be helpful in the overview at the beginning of \S2, to make it
  clearer to the reader where the manuscript is going.  It is also not
  entirely clear how the steps laid out there link to one another, and
  it would be useful to make this a bit more explicit.

  \textbf{TO DO!}

\item Some terms seem not be defined before first use.  It's not clear
  to me exactly what m\_0 and R\_10 each represent on page 3;
  ``transverse relaxation time'' is used on page 3 but without
  explanation of the meaning of ``transverse'' (likewise
  ``longitudinal''); again on page 3, it is not necessarily obvious
  which of S(t) and S(0) is pre-contrast, and which post-contrast; B1
  is used several times in \S2.3 but not defined; in \S3, pulse
  duration and separation are discussed but without any explanation of
  what these pulses are. I realise that many/most readers may know
  about these terms, but defining them, or avoiding their use entirely
  if they aren't really needed, would make the paper more accessible.

  \textbf{TO DO!}

\item The URL for ITK is given as ikg.org on page 4. This should be
  corrected.

  Fixed.

\item After equation (7), the paragraph begins, ``A long repetition
  time (TR $<=$ 5T1) ...''. Should this sign not be ``$>=$''?

  Fixed.

\item The discussion of anisotropic diffusion in \S3 seems odd, since
  equation (29) includes only a scalar diffusivity term, and therefore
  can represent only isotropic diffusion. Also in this section, the
  prescription to ignore measurements with b $<=$ 100 s/mm2 is a bit
  too general: the suitability of particular data will depend on the
  application. ``The diffusion of water ...'' (beginning of the third
  paragraph) should be ``The diffusivity of water ...''. Note also
  that diffusivity is temperature-dependent, so ``at room
  temperature'' or similar should be added.

  \textbf{TO DO!}

\item In the Bayesian estimation procedure, is there a particular
  reason for using the median rather than the mean of the posterior
  distribution as the summary statistic?

  The median is a more robust summary statistic and is frequently used
  as the point estimator in Bayesian inference.  The posterior median
  is prefered since it is typically closer to the posterior mode (the
  Bayesian analog to the maximum-likelihood estimator of location)
  than the posterior mean \citep{car-lou:Bayesian}.

\item In \S4.2, the authors state that ``... high ADC values correspond
  to pure isotrpic diffusion of water molecules in the tissue''.  This
  is misleading: isotropy and apparent/mean diffusivity are distinct
  characteristics.

  \textbf{TO DO!}

\item It's not clear to me what the authors mean by ``... the indirect
  nature of MRI data acquisition''.  Perhaps this could be clarified.

  \textbf{TO DO!}

\end{enumerate}

\section*{Reviewer \#2}

We would like to thank the reviewer for all his/her comments and
suggestions.

\subsection*{Comments}

\begin{enumerate}

\item P.1 - Why is it called dcemriS4 if it also does diffusion
  weighted image processing? DTI?

  \textbf{TO DO!}

\item P.2 – Many, many ``?'' question marks throughout the pdf
  file.. missing references, etc..

  This was a problem with the \texttt{Sweave} procedure used by the
  associate editor.  All citations/references have been checked and
  are valid in the manuscript.

\item P.2 – ``DWI quantifies the deviation of diffusion from
  isotropy'' This statement is not true..  DWI using equivalent
  strength gradients in the three orthogonal directions at varying
  strengths to measure the isotropic free diffusion path length in
  tissue. While technically the DWIx, DWIy, DWIz components could be
  examined individually, it is the trace of these that provides an
  isotropic path length mediated by cellular restrictions to the
  random Brownian motion.  i.e. The apparent diffusion coefficient
  (ADC) maps generated from a DWI acquisition contains no information
  about the degree or direction of anisotropy but will provide
  essential information on regions of infarct as well as boundaries
  and differentiation of cystic from malignant lesions.

  \textbf{TO DO!}

\item p.2 - ``Deviations from isotropy (anisotropic diffusion) in
  tissue is then used to infer biological information; e.g.,detection,
  disease progression, treatment response.''  Should be ``are then
  used to infer''... really only applies to white matter disease, i.e.
  degeneration or destruction of the fiber tracts..

  Changed.

\item p.2 – ``Typically, several structural sequences are performed
  (both T1- and T2-weighted) after the patient has been positioned in
  the scanner.'' - Not needed..

  Removed.

\item p.4 – ``For computational reasons, we follow the method of ?'' –
  Which method and why?

  \textbf{TO DO!}

\item Also, will you confirm what relaxation values you will be using
  for various Gadolinium chelates at differing field strengths.. in
  blood or plasma?

  \textbf{TO DO!}

\item P.5 – Should the actual command line R-code actually be included
  in the text or an appendix?

  We believe strongly that the \textsf{R} code should be interwoven
  with the text and not placed at the end of the manuscript in an
  appendix.

\item P.6 – ``By defining regions of interest (ROIs) in FSLView we may
  construct a mask that separates voxel belonging to the 10 unique
  gels.'' – should be ``separates voxels''

  Fixed.

\item P.7 – Need to reference Weinmann’s original paper on AIF's.

  I believe it is cited.

\item P.8 - Figure 3 - Why not just use a linear increase during bolus
  injection ($t<\tau$) (assuming that a power injector is used at a
  fixed rate) and then a single/bi-exponential afterwards?  This is
  done routinely in PET.  Should also show an actual AIF from a human
  subject and representative fit..

  \textbf{TO DO!}  

\item P.11 – ``To increase computational efficiency draws from the
  posterior distribution are implemented in C and linked to R.'' –
  What draws from the posterior distribution??

  As stated in the same paragraph, samples from the posterior PDF 
  are drawn. We replaced the sentence with ``To increase
  computational efficiency sampling from the posterior distribution is
  implemented in \textsf{C} and linked to \textsf{R}.''.

\item P.13 – ``Contrast is generated when the diffusion of molecules
  in tissue prefer a specific direction'' – Please see comment \#3 as
  DWI/ADC maps do not contain directional information.  Contrast may
  be obtained in the core infarct of an ischemic stroke in regions of
  isotropic diffusion.

  \textbf{TO DO!}  

\item P.13 – Might also mention higher diffusion b-values
  (i.e. $>1000$ s/mm2) to examine slow/fast components of diffusion
  related to intra/extracellular diffusion.

  \textbf{TO DO!}  

\item P.13 – Was even a paper out a few years back touching on
  diffusion in DCE analysis.  Pellerin M, Yankeelov TE, Lepage
  M. Incorporating contrast agent diffusion into the analysis of
  DCE-MRI data. Magn Reson Med. 2007 Dec;58(6):1124-34.

  \textbf{TO DO!}  

\item P.13 – ``Observing an increase in diffusivity..'' – Need to be
  cautious here as tumors may show increased ADC due to initial edemic
  response to therapy followed by decreased ADC as tumors begin to
  become necrotic or apototic. Additionally, tumor regression would be
  indicated by a reduction in ADC.. Lastly, ADC values in different
  tumor types do not all respond the same as well as variation in Tx
  (XRT, Chemo, etc.).

  \textbf{TO DO!}  

\item P.14 – ``we utilize a binary mask'' – How was this mask created
  and what threshold was decided upon to create the mask??

  \textbf{TO DO!}  

\item P.14 – ``from an appropriate voxel or collection of voxel'' –
  should be ``voxels''.

  Changed.

\item P.14 – Although selecting a literature based AIF may be
  appropriate for some neurological lesions, a population average
  closer to the feeding vessel of the tumor may be better served for
  other tumor types outside the brain.

  \textbf{TO DO!}  

\item P.15 – a reference to ``fritz.hansen'' is made in the
  code.. does this refer to a subject/colleague name?

  The name of this option refers to \citet{fri-etal:measurement}, who
  proposed values for the bio-exponential AIF.

\item P.15 – ``ve is high at the tumor rim'' - Is the maximum value of
  this parameter constrained? i.e. can there be $>$ 100\% for the
  extravascular extracellular space?

  \textbf{TO DO!}  

\item P.15 – Likewise is.. ve + vp $<$1 used as a constraint?

  \textbf{TO DO!}  

\item P.15 – ``The SSE over the given ROI covers a variety of tissue
  types'' – Has this program been tested or applied to non-cerebellar
  tumor types (i.e, breast, colorectal, prostate.)

  \textbf{TO DO!}  

\item P.21 – ``The methodology behind DWI and DTI are virtually
  identical.. so we will ignore the extra information provide''
  –should be ``provided''.  I understand the statement, but a few
  sentences about defining the maximum eigenvector of the diffusion
  ellipsoid with an figure might be informative for the
  reader.. although possibly beyond the scope of this article. It
  might show the orthogonal diameters on the ellipsoid used to
  calculated the DWI. (Might also mention that all major vendors
  compute the fractional anisotropy (FA) maps online.)

  \textbf{TO DO!}  

\item Also, in my experience, the ADC map created from the DTI data
  set may differ slightly from that acquired with a 3-Trace DWI
  sequence.

  \textbf{TO DO!}  

\item P.22 – ``range of physical units for the ADC values is [0.0005,
  0.003]'' – Interestingly, I have also seen values greater than this
  due to CSF flow phenomenon.

  \textbf{TO DO!}  

\item P.22 – ``isotrpic diffusion'' – should be ``isotropic''

  Changed.

\item In this reviewers opinion, the diffusion component of this
  package might not be included..  Calculation of apparent diffusion
  coefficient maps are now created online at the time of acquisition
  on most major MRI vendors.  In addition, the creation of the maps is
  the result of a straight linear regression which is must less
  involved than the DCE-MRI analysis presented in this work.

  \textbf{TO DO!}  

\item P.26 – Are there capabilities available for incorporation of a
  user-defined AIF taken from the DCE-MRI data set??

  Yes, it is possible to supply a user-defined AIF.  We have added a
  few sentences in the RIDER~Neuro~MRI example that highlights this
  feature.

\end{enumerate}

%--------------------------------------------------------------------%

\bibliography{dcemri}

%--------------------------------------------------------------------%

\end{document}
