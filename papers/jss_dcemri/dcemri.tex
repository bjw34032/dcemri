\documentclass[article]{jss}

%%%%%%%%%%%%%%%%%%%%%%%%%%%%%%
%% declarations for jss.cls %%%%%%%%%%%%%%%%%%%%%%%%%%%%%%%%%%%%%%%%%%
%%%%%%%%%%%%%%%%%%%%%%%%%%%%%%

\usepackage{amsmath}

\newcommand{\ktrans}{K^\text{trans}}
\newcommand{\iaugc}[1]{\text{IAUGC}_{#1}}
\newcommand{\kep}{k_\text{ep}}
\newcommand{\vp}{v_\text{p}}

%% almost as usual
\author{Brandon Whitcher\\GlaxoSmithKline \And 
        Volker J. Schmid\\Ludwig-Maximilians Universit\"at M\"unchen}
\title{Quantitative Analysis of Dynamic Contrast-enhanced {MRI} ({DCE-MRI}) and Diffusion-weighted {MRI} ({DW-MRI}) for Oncology Studies in \proglang{R}}

%% for pretty printing and a nice hypersummary also set:
\Plainauthor{Brandon Whitcher, Volker Schmid} %% comma-separated
\Plaintitle{Quantitative Analysis of Dynamic contrast-Enhanced {MRI} ({DCE-MRI}) and Diffusion-weighted {MRI} ({DW-MRI}) for Oncology Studies in R} %% without formatting
\Shorttitle{DCE-MRI and DW-MRI in R} %% a short title (if necessary)

%% an abstract and keywords
\Abstract{
  The abstract of the article.
}
\Keywords{contrast, \pkg{dcemriS4noNifti}, diffusion, dynamic, enhanced, imaging, magnetic, resonance}
\Plainkeywords{contrast, dcemriS4noNifti, diffusion, dynamic, enhanced, imaging, magnetic, resonance} %% without formatting
%% at least one keyword must be supplied

%% publication information
%% NOTE: Typically, this can be left commented and will be filled out by the technical editor
%% \Volume{13}
%% \Issue{9}
%% \Month{September}
%% \Year{2004}
%% \Submitdate{2004-09-29}
%% \Acceptdate{2004-09-29}

%% The address of (at least) one author should be given
%% in the following format:
\Address{
  Brandon Whitcher\\
  GlaxoSmithKline\\
  Clinical Imaging Centre\\
  Hammersmith Hospital\\
  Du Cane Road\\
  London W12 0HS, United Kingdom\\
  E-mail: \email{bjw34032@users.sourceforge.net}\\
  URL: \url{http://www.dcemri.org/}
}
%% It is also possible to add a telephone and fax number
%% before the e-mail in the following format:
%% Telephone: +43/1/31336-5053
%% Fax: +43/1/31336-734

%% for those who use Sweave please include the following line (with % symbols):
%% need no \usepackage{Sweave.sty}

%% end of declarations %%%%%%%%%%%%%%%%%%%%%%%%%%%%%%%%%%%%%%%%%%%%%%%


\begin{document}

%% include your article here, just as usual
%% Note that you should use the \pkg{}, \proglang{} and \code{} commands.

\section[Introduction]{Introduction}
%% Note: If there is markup in \(sub)section, then it has to be escape as above.

Quantitative analysis of perfusion imaging using dynamic
contrast-enhanced MRI (DCE-MRI) is achieved through a series of
processing steps, starting with the raw data acquired from the MRI
scanner, and involves a combination of physics, mathematics,
engineering and statistics.  The purpose of the \pkg{dcemriS4noNifti} package
is to provide a collection of functions that move the experimental
data through all steps of the data analysis pipeline using standard
data formats that may be visualized and manipulated across a wide
variety of software packages.  

\section[Dynamic Contrast-enhanced Magnetic Resonance Imaging]{Dynamic Contrast-enhanced Magnetic Resonance Imaging}

\subsection[Motion Correction and Co-registration]{Motion Correction and Co-registration}

Basic motion correction within an acquisition, and co-registration
between acquired series, is available using template matching
\citep{lew:template-matching}.  A reference volume must be
pre-specified where a mask has been applied to remove all voxels that
should not be included in the algorithm.  Note, only three-dimensional
translations are allowed and no interpolation is used (i.e., only
whole-voxel translations) at this time.

\subsection[T1 Relaxation and Gadolinium Concentration]{T1 Relaxation and Gadolinium Concentration}

Estimation of the tissue T1 relaxation rate is the first step in
converting signal intensity, obtained in the dynamic acquisition of
the DCE-MRI protocol, to contrast agent concentration.  The subsequent
steps provided here focus on pharmacokinetic modeling and assumes one
has converted the dynamic acquisition to contrast agent concentration.
Please see \cite{col-pad:ieee} for a discussion on this point.

There are a myriad of techniques to quantify T1 using MRI.  Currently
curve-fitting methods for two popular acquisition schemes are
available
\begin{itemize}
\item Inversion recovery (\url{www.e-mri.org/mri-sequences/inversion-recovery-stir-flair.html})
\item Multiple flip angles \citep{par-pad:DCE-MRI}
\end{itemize}
Once the tissue T1 relaxation rate has been estimated, the dynamic
acquisition is then converted to contrast agent concentration.  Note,
the B1 field is assumed to be constant (and accurate) when using
multiple flip angles to estimate T1.  At higher fields strengths
(e.g., 3T) the B1 field should be estimated in order to correct the
prescribed flip angles.

\subsection{B1 Mapping Via the Saturated Double-Angle Method}

For in vivo MRI at high field ($\geq{3}$Tesla) it is essential to
consider the homogeneity of the active B1 field (B1+).  The B1+ field
is the transverse, circularly polarized component of B1 that is
rotating in the same sense as the magnetization.  When exciting or
manipulating large collections of spins, non-uniformity in B1+ results
in nonuniform treatment of spins.  This leads to spatially varying
image signal and image contrast and to difficulty in image
interpretation and image-based quantification
\citep{cun-pau-nay:saturated}.

The proposed method uses an adaptation of the double angle method
(DAM).  Such methods allow calculation of a flip-angle map, which is
an indirect measure of the B1+ field.  Two images are acquired:
$I_1$ with prescribed tip $\alpha_1$ and $I_2$ with prescribed tip
$\alpha_2=2\alpha_1$.  All other signal-affecting sequence parameters
are kept constant. For each voxel, the ratio of magnitude images
satisfies
\begin{equation*}
  \frac{I_2(r)}{I_1(r)} =
  \frac{\sin\alpha_2(r)f_2(T_1,\text{TR})}{\sin\alpha_1(r)f_1(T_1,\text{TR})}
\end{equation*}
where $r$ represents spatial position and $\alpha_1(r)$ and
$\alpha_2(r)$ are tip angles that vary with the spatially varying B1+
field.  If the effects of $T_1$ and $T_2$ relaxation can be neglected,
then the actual tip angles as a function of spatial position satisfy
\begin{equation*}
  \alpha(r) = \text{arccos}\left(\left|\frac{I_2(r)}{2I_1(r)}\right|\right)
\end{equation*}
A long repetition time ($\text{TR}\leq{5T_1}$) is typically used with
the double-angle methods so that there is no $T_1$ dependence in
either $I_1$ or $I_2$ (i.e.,
$f_1(T_1,\text{TR})=f_2(T_1,\text{TR})=1.0$.  Instead, the proposed
method includes a magnetization-reset sequence after each data
acquisition with the goal of putting the spin population in the same
state regardless of whether the or $\alpha_2$ excitation was used for
the preceding acquisition (i.e.,
$f_1(T_1,\text{TR})=f_2(T_1,\text{TR})\ne{1.0}$).

\subsubsection{Example}
Using data acquired from a T1 phantom at two flip angles,
$\alpha_1=60^\circ$ and $\alpha_2=120^\circ$, we compute the
multiplicative factor relative to the low flip angle using the
saturated double-angle method \citep{cun-pau-nay:saturated}.
  
\begin{Schunk}
\begin{Sinput}
R> sdam60 <- readNIfTI(system.file("nifti/SDAM_ep2d_60deg_26slc.nii.gz", 
+     package = "dcemriS4noNifti"))
R> sdam120 <- readNIfTI(system.file("nifti/SDAM_ep2d_120deg_26slc.nii.gz", 
+     package = "dcemriS4noNifti"))
R> sdam.image <- rowMeans(dam(sdam60, sdam120, 60), dims = 3)
R> mask <- (rowSums(sdam60, dims = 3) > 500)
\end{Sinput}
\end{Schunk}
\begin{Schunk}
\begin{Sinput}
R> SDAM <- readNIfTI(system.file("nifti/SDAM_smooth.nii.gz", package = "dcemriS4noNifti"))
R> overlay(sdam120, ifelse(mask, SDAM, NA), z = 13, zlim.x = range(sdam120), 
+     zlim.y = c(0.5, 1.5), plot.type = "single")
\end{Sinput}
\end{Schunk}

Figure~\ref{fig:sdam} is the estimated B1+ field (with isotropic
Gaussian smoothing) for a gel-based phantom containing a variety of T1
relaxation times.  The center of the phantom exhibits a flip angle
$>60^\circ$ while the flip angle rapidly becomes $<60^\circ$ when
moving away from the center in either the $x$, $y$ or $z$ dimensions.
Isotropic smoothing should be applied before using this field to
modify flip angles associated with additional acquisitions; e.g., in
the \pkg{AnalyzeFMRI} package \citep{AnalyzeFMRI}.
  
\begin{figure}[!htbp]
  \centering
  \includegraphics*[width=.5\textwidth]{sdam.png}
  \caption{Estimated B1+ field (with isotropic Gaussian smoothing)
    using the saturated double-angle method.  The colors correspond to
    a multiplicative factor relative to the true flip angle
    ($60^\circ$).}
  \label{fig:sdam}
\end{figure}

Assuming the smoothed version of the B1+ field has been computed
(\code{SDAM}), multiple flip-angle acquisitions can be used to estimate
the T1 relaxation rate from the subject (or phantom).  The
multiplicative factor, derived from the saturated double-angle method,
is used to produce a spatially-varying flip-angle map and input into
the appropriate function.

\begin{Schunk}
\begin{Sinput}
R> alpha <- c(5, 10, 20, 25, 15)
R> nangles <- length(alpha)
R> fnames <- paste("fl3d_vibe-", alpha, "deg.nii.gz", sep = "")
R> X <- Y <- 64
R> Z <- 36
R> flip <- fangles <- array(0, c(X, Y, Z, nangles))
R> for (w in 1:nangles) {
+     vibe <- readNIfTI(system.file(paste("nifti", fnames[w], sep = "/"), 
+         package = "dcemriS4noNifti"))
+     flip[, , 1:nsli(vibe), w] <- vibe
+     fangles[, , , w] <- array(alpha[w], c(X, Y, Z))
+ }
R> TR <- 4.22/1000
R> fanglesB1 <- fangles * array(SDAM, c(X, Y, Z, nangles))
R> zi <- 10:13
R> maskzi <- mask
R> maskzi[, , (!1:Z %in% zi)] <- FALSE
R> R1 <- R1.fast(flip, maskzi, fanglesB1, TR, verbose = TRUE)
\end{Sinput}
\begin{Soutput}
  Deconstructing data...
  Calculating R10 and M0...
  Reconstructing results...
\end{Soutput}
\end{Schunk}
\begin{Schunk}
\begin{Sinput}
R> overlay(vibe, 1/R1$R10[, , 1:nsli(vibe)], z = 13, zlim.x = c(0, 
+     1024), zlim.y = c(0, 2.5), plot.type = "single")
\end{Sinput}
\end{Schunk}

Figure~\ref{fig:t1-phantom} displays the quantitative T1 map for a
gel-based phantom using information from the estimated B1+ field.

\begin{figure}[!htbp]
  \centering
  \includegraphics*[width=.5\textwidth]{t1_phantom.png}
  \caption{Estimated T1 relaxation rates for the phantom data
  acquisition.  The colors range from 0-2.5 seconds.}
  \label{fig:t1-phantom}
\end{figure}

By defining regions of interest (ROIs) in 

\begin{Schunk}
\begin{Sinput}
R> t1pmask <- readNIfTI(system.file("nifti/t1_phantom_mask.nii.gz", 
+     package = "dcemriS4noNifti"))
R> pmask <- nifti(array(t1pmask[, , 25], dim(t1pmask)))
\end{Sinput}
\end{Schunk}

\begin{figure}[!htbp]
\begin{center}
\begin{Schunk}
\begin{Sinput}
R> T1 <- c(0.484, 0.35, 1.07, 0.648, 0.456, 1.07, 0.66, 1.543, 1.543, 
+     0.353)
R> par(mfrow = c(1, 1), mar = c(5, 4, 4, 2) + 0.1)
R> boxplot(split(1/drop(R1$R10), as.factor(drop(pmask)))[-1], ylim = c(0, 
+     2.5), xlab = "Region of Interest", ylab = "T1 (seconds)")
R> points(1:10, T1, col = rainbow(10), pch = 16, cex = 2)
\end{Sinput}
\end{Schunk}
\includegraphics{dcemri-figure7}
\end{center}
\caption{Boxplots of the estimated T1 values for the gel-based
  phantom, grouped by user-specified regions of interest.  True T1
  values are plotted as colored circles for each distinct ROI.}
\label{fig:t1-phantom-boxplot}
\end{figure}

We may compare the ``true'' T1 values for each ROI with those obtained
from acquiring multiple flip angles with the application of B1
mapping.  Figure~\ref{fig:t1-phantom-boxplot} compares T1 estimates in
the 10 ROIs, defined by \code{pmask}, with the true T1 values (large
circles).  The first seven ROIs correspond to the cylinders that run
around the phantom, clockwise starting from approximately one o'clock.
The eighth and ninth ROIs are taken from the main compartment in the
phantom; ROI eight is drawn in the middle of the phantom while ROI
nine is drawn from the outside of the phantom.  The final ROI is taken
from the central cylinder embedded in the phantom.

\subsection{Contrast Agent Concentration}

The \code{CA.fast} function rearranges the assumed multidimensional
(2D or 3D) structure of the multiple flip-angle data into a single
matrix to take advantage of internal R functions instead of loops, and
called \code{E10.lm}.  Conversion of the dynamic signal intensities to
contrast agent concentration is performed via
\begin{equation*}
  [\text{Gd}] = \frac{1}{r_1}\left(\frac{1}{T_1} - \frac{1}{T_{10}}\right),
\end{equation*}
where $r_1$ is the spin-lattice relaxivity constant and $T_{10}$ is
the spin-lattice relaxation time in the absence of contrast media
\citep{buc-par:measuring}.  For computational reasons, we follow the
method of \cite{li-etal:improved}.

\subsection{Arterial Input Function}

Whereas quantitative PET studies routinely perform arterial
cannulation on the subject in order to characterize the arterial input
function (AIF), it has been common to use literature-based AIFs in the
DCE-MRI literature.  Examples include
\begin{equation*}
  C_p(t) = D \left( a_1 e^{-m_1t} + a_2 e^{-m_2t} \right),
\end{equation*}
where $D=0.1\,\text{mmol/kg}$, $a_1=3.99\,\text{kg/l}$,
$a_2=4.78\,\text{kg/l}$, $m_1=0.144\,\text{min}^{-1}$ and
$m_2=0.0111\,\text{min}^{-1}$
\citep{wei-lan-mut:pharmacokinetics,tof-ker:measurement}; or
$D=1.0\,\text{mmol/kg}$, $a_1=2.4\,\text{kg/l}$,
$a_2=0.62\,\text{kg/l}$, $m_1=3.0$ and $m_2=0.016$
\citep{fri-etal:measurement}.  There has been progress in measuring
the AIF using the dynamic acquisition and fitting a parametric model
to the observed data.  Recent models include a mixture of Gaussians
\citep{par-etal:derived} and sums of exponentials
\citep{ort-etal:efficient}.  \pkg{dcemriS4noNifti} has incorporated
one of these parametric models 
\begin{equation*}
  C_p(t) = A_B t e^{-\mu_Bt} + A_G \left( e^{-\mu_Gt} + e^{-\mu_Bt}\right)
\end{equation*}
\citep{ort-etal:efficient}, which can be fitted to the observed data
using nonlinear regression.  Using the AIF defined in
\cite{buc:uncertainty}, we illustrate fitting a parametric model to
characterize observed data.  The \code{orton.exp.lm} function provides
this capability using a common double-exponential parametric form.

\begin{Schunk}
\begin{Sinput}
R> data("buckley")
R> aifparams <- with(buckley, orton.exp.lm(time.min, input))
R> fit.aif <- with(aifparams, aif.orton.exp(buckley$time.min, AB, muB, 
+     AG, muG))
\end{Sinput}
\end{Schunk}
\begin{figure}[!htbp]
\begin{center}
\begin{Schunk}
\begin{Sinput}
R> with(buckley, plot(time.min, input, type = "l", lwd = 2, xlab = "Time (minutes)", 
+     ylab = ""))
R> with(buckley, lines(time.min, fit.aif, lwd = 2, col = 2))
R> legend("topright", c("Simulated AIF", "Estimated AIF"), lwd = 2, 
+     col = 1:2, bty = "n")
\end{Sinput}
\end{Schunk}
\includegraphics{dcemri-figure8}
\end{center}
\caption{Arterial input function (AIF) from \cite{buc:uncertainty} and
  the best parametric fit, using the exponential model from
  \cite{ort-etal:efficient}.}
\label{fig:fitted-aif}
\end{figure}

Figure~\ref{fig:fitted-aif} shows both the true AIF and the best
parametric description using a least-squares fitting criterion.

\subsection[Kinetic Parameter Estimation]{Kinetic Parameter Estimation}

The standard Kety model \citep{ket:blood-tissue}, a single-compartment
model, or the extended Kety model, the standard Kety model with an
extra ``vascular'' term, form the collection of basic parametric
models one can apply using \pkg{dcemri}.  Regardless of which
parametric model is chosen for the biological system, the contrast
agent concentration curve at each voxel in the region of interest
(ROI) is approximated using the convolution of an arterial input
function (AIF) and the compartmental model; e.g.,
\begin{eqnarray*}
  C_t(t) &=& \ktrans \left[ C_p(t) \otimes e^{-\kep t} \right],\\
  C_t(t) &=& \vp C_P(t) + \ktrans \left[ C_p(t) \otimes e^{-\kep t}
    \right].
\end{eqnarray*}

Parameter estimation is achieved using one of two options in the
current version of this software:
\begin{itemize}
\item Non-linear regression using non-linear least squares
  (Levenburg-Marquardt optimization)
\item Bayesian estimation using Markov chain Monte Carlo (MCMC)
  \citep{sch-etal:TMI}
\end{itemize}
Least-square estimates of the kinetic parameters $\ktrans$ and $\kep$
(also $\vp$ for the extended Kety model) are provided in
\code{dcemri.lm} while the posterior median is provided in
\code{dcemri.bayes}.  When using Bayesian estimation all samples from
the joint posterior distribution are also provided, allowing one to
interrogate the empirical probability density function (PDF) of the
parameter estimates.

Using the simulated breast data from \cite{buc:uncertainty}, we
illustrate fitting the ``extended Kety'' model to the contrast agent
concentration curves using the exponential model for the AIF.  We use
non-linear regression to fit the data on an under-sampled subset (in
time) of
the simulated curves.

\begin{Schunk}
\begin{Sinput}
R> xi <- seq(5, 300, by = 5)
R> img <- array(t(breast$data)[, xi], c(13, 1, 1, 60))
R> time <- buckley$time.min[xi]
R> aif <- buckley$input[xi]
R> mask <- array(TRUE, dim(img)[1:3])
R> aifparams <- orton.exp.lm(time, aif)
R> fit <- dcemri.lm(img, time, mask, model = "orton.exp", aif = "user", 
+     user = aifparams)
\end{Sinput}
\end{Schunk}
\begin{figure}[!htbp]
\begin{center}
\begin{Schunk}
\begin{Sinput}
R> par(mfrow = c(4, 4), mar = c(5, 4, 4, 2)/1.25, mex = 0.8)
R> for (x in 1:nrow(img)) {
+     plot(time, img[x, 1, 1, ], ylim = range(img), xlab = "Time (minutes)", 
+         ylab = "", main = paste("Series", x))
+     kinparams <- with(fit, c(vp[x, 1, 1], ktrans[x, 1, 1], kep[x, 
+         1, 1]))
+     lines(time, model.orton.exp(time, aifparams[1:4], kinparams), 
+         lwd = 1.5, col = 2)
+ }
\end{Sinput}
\end{Schunk}
\includegraphics{dcemri-figure9}
\end{center}
\caption{Simulated signal intensity curves from Buckley (2002), for
  breast tissue, with the best parametric fit using an exponential
  model for the AIF and the ``extended Kety'' model.}
\label{fig:fitted-kinetic}
\end{figure}

Figure~\ref{fig:fitted-kinetic} displays the 13 unique simulated
curves along with the fitted curves from the compartmental model.
There is decent agreement between the observed and fitted values,
except for Series~6 which changes too rapidly in the beginning and
cannot be explained by the parametric model.

\subsection[Statistical Inference]{Statistical Inference}

No specific support is provided for hypothesis testing in
\pkg{dcemriS4noNifti}.  We recommend one uses built-in facilities in
\proglang{R} to perform ANOVA (analysis of variance) or mixed-effects
models based on statistical summaries of the kinetic parameters over
the ROI per subject per visit.  An alternative to this traditional
approach is to analyze an entire study using a Bayesian hierarchical
model \citep{sch-etal:hierarchical}, available in the software project
\pkg{PILFER} (\url{pilfer.sourceforge.net}).

One may also question the rationale for hypothesis testing in only one
kinetic parameter.  Preliminary work has been performed in looking at
the joint response to treatment of both $\ktrans$ and $\kep$ in
DCE-MRI by \cite{oco-etal:fPCA}.

\section[Diffusion Weighted Imaging]{Diffusion Weighted Imaging}

Diffusion weighted imaging (DWI), also known as diffusion-weighted MRI
(DW-MRI), is a technique that measures the Brownian motion of water
molecules to estimate the diffusion characteristics of tissue \emph{in
vivo} \citep{mos-etal:diffusion,bux:introduction}.  Contrast is
generated when the diffusion of molecules in tissue prefer a specific
direction in three-dimenisonal space (anisotropic diffusion) relative
to any particular direction (isotropic diffusion).  Using the
Stejskal--Tanner equation
\begin{equation*}
  \frac{S}{S_0} =
  \exp\left(-\gamma^2G^2\delta^2(\Delta-\delta/3)D\right) =
  \exp\left(-bD\right),
\end{equation*}
one may solve for the unknown diffusion to obtain the \emph{apparent
diffusion coefficient} (ADC) $D$.  For completeness, $S_0$ is the
signal intensity without the diffusion weighting, $S$ is the signal
with the gradient applied, $\gamma$ is the gyromagnetic ratio, $G$ is
the strength of the gradient pulse, $\delta$ is the duration of the
gradient pulse and $\Delta$ is the time between the two pulses.  The
functions \code{ADC.fast} and \code{adc.lm} perform parameter
estimation using a similar interface to kinetic parameter estimation
previously introduced for DCE-MRI.

DWI is under rapid development as an oncology imaging biomarker
\citep{che-etal:diffusion,koh-col:DW-MRI}.  The diffusion of water
without restrictions is about
$3.0{\times}10^{-3}\;\text{mm}^2/\text{s}$.  Once the ADC is estimated
in the tumor of interest at baseline, treatment response may be
assessed at subsequent time points.  The most appropriate timings
depend on both the type of tumor and treatment regime.  Observing an
decrease in diffusivity, via a decrease in the ADC values
post-treatment, may be a result of cell swelling after ititial
chemotherapy or radiotherapy followed by an increase in diffusivity,
via an increase in the ADC values, from cell necrosis and lysis.  A
decrease in ADC values may be observed directly through tumor
apoptosis after treatment.  

\section*{Acknowledgments}

The authors would like to thank...

\bibliography{dcemri}

\end{document}
















